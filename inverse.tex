\documentclass{article}
\usepackage{amsmath}
\setlength{\parindent}{0pt}

\begin{document}
Inverse algorithm based on the double dual form of a vector space.

Let a vector space be formed as the type:
\[ V a :: a \to R \to R \]

This data type has a natural monad instance. Linear maps are a sub type of:
\[ LM a b :: V a \to V b \]

In fact, we can use the bind instance to construct linear maps as:
\begin{align*}
    extend &:: (Monad m) => (a \to m b) \to m a \to m b \\
    extend &= flip (>>=) 
\end{align*}
In particular, we have that:
\[  extend :: (a \to V b) \to LM a b \]

To extract an othogonal map out of a linear map, we need to look for a
off diagonal aspect, and add extract out a corresponding rotation. 

Example:
Let $A$ and $B$ be finite sets of size $n$ and $m$ respectively:
\begin{align*}
    A &= \{a_i \mid i \in 1 .. n \} \\
    B &= \{b_i \mid i \in 1 .. m \} \\
\end{align*}

Then a linear map $l \in LM A B$ is given by:
\[ l(a_i) = \sum_j l_i^j b_j \]

What we want to do is align the linear map to the basis on $B$. In this way
decompose $l$ into $ l = d \circ o$ where $o : V A \to V A$ is orthogonal and $d
: V A \to V B $ is diagonal.

For any element $x \in V b$ we can calculate a couple of useful functions:
\begin{align*}
   \mathrm{mix}\ x  &= \{ b \mid b \in B, x\ \delta_b > 0 \} \\
   \mathrm{size}\ x &= \| \mathrm{mix}\ x \|
\end{align*}

Now we can define what it means for $d :: V a \to V b$ to be diagonal: $d$ is diagonal if 
\begin{itemize}
    \item $\forall a \in A : \mathrm{size}\ d a = 1$
    \item $\forall a,a' \in A: \mathrm{mix}\ d a \cap \mathrm{mix}\ d a' = \phi $
\end{itemize}

% let l :: V A \to V B ( :: LM A B )
% let alreadyDiagonal = [] 
% For each a in A:
%   compute: bs = coef ( l a )
%   if length bs = 1
%   then for (b', c) in bs:
%     if b' in alreadyDiagonal
%       then do 
%              r = createrotation a c'
%              alreadyDiagonal add b'
  
\end{document}
